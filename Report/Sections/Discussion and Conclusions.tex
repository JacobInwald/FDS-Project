\section{Discussion and Conclusions}
    % Suggested 400 words.
    \begin{multicols}{2}
        \paragraph{Summary of findings}
            Critics play a major role in the movie industry.
            They are seen as weilding enormous power, with the ability to render years of work wasted with a
                single review.
            We aimed to investigate if critics truly have this power.
            We found that critics can't predict box office success, bringing into question their power over movie
                revenue.
            However, we did find that critics can predict the general success and reception of a movie.
            Their ratings have a high correlation with user ratings, and our multiple regression model used Metascore
                as a predictor for success.
            These findings suggest that while critics can predict the success of a movie, they don't appear to impact
                the revenue the movie will bring in.

            We also found that the general public is more lenient than critics.
            The critic scores follow a unskewed normal distribution, whereas user ratings is left-skewed (see fig.~\ref{fig-distribution-of-numeric-variable}).
            This means that on average the general public rates movies higher than critics will.
            These findings suggest that critics provide more unbiased reviews of movies than a consumer will.

            We also found that actor experience is a better predictor of box office success than the director experience.
            This was reinforced by its moderate correlation to box office success, which is inline with research 
                about "star power", the idea that stars drive movie success \cite{bibid}.
            
        \paragraph{Evaluation of own work: strengths and limitations}
            One limitation of our approach was the small amount of data we worked with.
            The main dataset we used - the IMDb dataset - was just a sample of a larger
                database and had only 1000 entries.
            An improvement we could make would be to purchase the full dataset and run our
                analysis on that dataset.
            Another limitation was the 
            
        \paragraph{Comparison with any other related work}
            E.g. ``Anscombe has also demonstrated that many patterns of data can have the
                same correlation coefficient''.

            Wikipedia can also be cited but it is better if you find the original reference
                it for a particular claim in the list of references on the Wikipedia page, read
                it, and cite it.

            The golden rule is always to cite information that has come from other sources,
                to avoid plagiarism.

        \paragraph{Improvements and extensions}


    \end{multicols}