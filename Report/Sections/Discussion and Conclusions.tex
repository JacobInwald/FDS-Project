\section{Discussion and Conclusions}
    % Suggested 400 words.
    \begin{multicols}{2}
        \paragraph{Summary of findings}
            Critics play a major role in the movie industry.
            They are seen as wielding enormous power, with the ability to render years of
                work wasted with a single review.
            We aimed to investigate if critics truly have this power.
            We found that critics can't predict box office success, bringing into question
                their power over movie revenue.
            However, we did find that critics can predict the general success and reception
                of a movie.
            Their ratings have a high correlation with user ratings, and our multiple
                regression model used Metascore as a predictor for success.
            These findings suggest that while critics can predict the success of a movie,
                they don't appear to impact the revenue the movie will bring in.
            Moreover, we found that the general public is more lenient than critics.
            The critic scores follow a unskewed normal distribution, whereas user ratings
                is left-skewed (see fig.~\ref{fig-distribution-of-numeric-variable}).
            This means that on average the general public rates movies higher than critics
                will.
            These findings suggest that critics provide a more unbiased review of movies
                than a consumer does.
            We also found that actor experience is a better predictor of box office success
                than the director experience.
            This was reinforced by its moderate correlation to box office success, which is
                inline with research about "star power", the idea that stars drive movie
                success~\cite{elberse2007power}.

        \paragraph{Evaluation of own work}
            As a way to factor for non-linearity in the data, we normalised and
                standardised the data.
            We tested our models by checking that they maintained similar $R^2$ values for
                train and test data and that the residuals verified linearity of data.
            One limitation of our approach was the small amount of data we worked with.
            The main dataset we used - the IMDb dataset - was just a sample of a larger
                database and had only 1000 entries.
            Another limitation was the lack of some would be useful columns from the
                dataset.
            One such example is budget, which was used in Ahmad et
                al.~\cite{ahmadDuraisamyYousefBuckles} to produce a model for predicting
                revenue of a movie.

        \paragraph{Comparison with any other related work}
            Elberse~\cite{elberse2007power} has also demonstrated that actors influence the
                box office success of movies, also attributing this to lead actor experience.
            Dhir and Raj~\cite{dhirRaj} also created a model to predict success, defining
                it in this case as user rating.
            Ahmad et al.~\cite{ahmadDuraisamyYousefBuckles} also created a model to predict
                success, defining it in this case as revenue generated.

        \paragraph{Improvements and extensions}
            An improvement we could make would be to purchase the full dataset and run our
                analysis on that dataset.
            Another improvement would be to find a dataset that combines various other
                aspects of movie creation, for example movie budget or production managers.

    \end{multicols}
