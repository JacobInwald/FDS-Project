\section{Data}
    % Suggested 300 words

    \paragraph{Data provenance}
        We used two datasets to aid our investigation: an IMDB movie dataset, containing
            various details about movies made from 2006-2016; and a larger movie dataset,
            containing details about movies released on or before July 2017.
        The IMDB dataset has a bit of a contentious past as it was scraped off of the movie
            ranking website IMDB.com, and is actually only a sample dataset from a much larger
            dataset that has every movie made from 2006-2016 that is in the IMDB.
        The large movie dataset was collated using data from TMDB (The Movie DataBase) and 
            grouplens.org; but we only use the part that was obtained from TMDB as it is used
            to find Director and Actor experience.

        Both datasets were downloaded from kaggle.com, a data science platform that enables users
            to access and share datasets. These datsets are shared underneath the CC0 1.0 Universal
            Public Domain Dedication, as such we will use them underneath fair use.

    \paragraph{Data description}
        The IMDB movie dataset has 12 columns, all either containing string values or floating point values.
        The only odd column is the Genre column which contains the different genres that can be applied to a particular movie
        The genres that are in this dataset are the arbritary genres: 
        \begin{multicols}{5}
            \begin{itemize}
                \item Action 
                \item Adventure 
                \item Sci-Fi 
                \item Mystery 
                \item Horror 
                \item Thriller 
                \item Animation 
                \item Comedy 
                \item Family 
                \item Fantasy 
                \item Drama 
                \item Music 
                \item Biography 
                \item Romance 
                \item History 
                \item Crime 
                \item Western 
                \item War 
                \item Musical 
                \item Sport
            \end{itemize}
        \end{multicols}
        \begin{figure}[h]
            \centering
            \begin{tabular}{ll}
                \toprule
                Column Name &           Description \\
                \midrule
                Rank &                  The rank the movie has in the IMDB database \\
                Title &                 The name of the movie \\
                Genre &                 The genres that apply to the movie, there can be anywhere from 1-3 genres. \\
                {}      &               A genre can be any from: Action, Adventure, Sci-Fi, Mystery, Horror,       \\
                {}      &               Thriller, Animation, Comedy, Family, Fantasy, Drama, Music, Biography,     \\
                {}      &               Romance, History, Crime, Western, War, Musical, Sport                      \\
                Description &           The description of the movie \\
                Director &              The person who directed the movie \\
                Actors &                The lead roles in the movie \\
                Year  &                 The year the movie was released \\
                Runtime (Minutes) &     The runtime in minutes of the movie \\ 
                Rating   &              The mean rating of the movie, taken from IMDB.com \\
                Votes   &               The amount of users that voted on a movie to give it that rating \\ 
                Revenue (Millions) &    The gross income the movie made at the US box office\\  
                Metascore   &           The rating of movie, determined using aggregated weighted critic scores \\ 
                \bottomrule
            \end{tabular}
            \caption[short]{The different columns in the IMDB-Movie-Data.csv file}\label{fig-IMDB-Movie-Data-Column-Description}
        \end{figure}
        The other dataset used had a bit of a more odd structure.
        We used the crew.csv file obtained from 


    \paragraph{Data processing}
        How you have processed the dataset, e.g., cleaning, removing missing values,
            joining tables.
