\section{Data}
    % Suggested 300 words

    \paragraph{Data provenance}
        We utilized two movie datasets: an IMDb dataset\cite{data:IMDb} and
            TMD\cite{data:TMD} (The Movie Dataset).
        The IMDb dataset contains data about movies made from 2006-2016, while the TMD
            dataset contains data about movies released on or before July 2017.
        Both datasets were obtained from kaggle.com and are shared under the CC0 1.0
            Universal Public Domain Dedication.
        TMD is merged data from TMDb (The Movie DataBase) and grouplens.org, a movie
            ranking site.
        However, we only used the part that was obtained from TMDb.
        It is worth noting that the provenance of the IMDb dataset has been subject to
            controversy as it was scraped from IMDb.com.
        However, for this investigation, we used only a sample of this dataset which is
            publicly available on kaggle.com.

    \paragraph{Data description}
        The IMDb movie dataset contains 12 columns consisting of string or floating
            point values.
        One unique column is the Genre, which lists the arbitrary genres assigned to
            each movie in the dataset such as Action, Adventure, Sci-Fi, Mystery, Horror,
            Thriller, Animation, Comedy, Family, Fantasy, Drama, Music, Biography, Romance,
            History, Crime, Western, War, Musical and Sport.
        The column summary is shown in Table
            \ref{tab-IMDb-Movie-Data-Column-Description}.
        \begin{table}[h]
            \centering
            \begin{tabular}{lp{10cm}l}
                \toprule
                Column Name        & Description                                                                & Data Type \\
                \midrule
                Rank               & The rank the movie has in the IMDb database                                & Integer   \\
                Title              & The name of the movie                                                      & String    \\
                Genre              & The genres that apply to the movie, there can be anywhere from 1-3 genres.
                A genre can be any from: Action, Adventure, Sci-Fi, Thriller, Animation,
                    Comedy, Family, Fantasy, Drama, Music, Romance, History, Crime, Western, War,
                    Musical, Sport, Horror, Mystery, Biography.
                                   & Genre                                                                                  \\
                Description        & The description of the movie                                               & String    \\
                Director           & The person who directed the movie                                          & String    \\
                Actors             & The lead roles in the movie                                                & String    \\
                Year               & The year the movie was released                                            & Integer   \\
                Runtime (Minutes)  & The runtime in minutes of the movie                                        & Integer   \\
                Rating             & The mean rating of the movie, taken from IMDb.com                          & Float     \\
                Votes              & The amount of users that voted on a movie to give it that rating           & Integer   \\
                Revenue (Millions) & The gross income the movie made at the US box office                       & Float     \\
                Metascore          & The rating of movie, determined using aggregated weighted Critics scores   & Integer   \\
                \bottomrule
            \end{tabular}
            \caption[short]{The different columns in the IMDb-Movie-Data.csv file}\label{tab-IMDb-Movie-Data-Column-Description}
        \end{table}

        We only worked with the crew table in TMD, which is a small part of the whole
            database.
        This table has three columns - cast, crew, and id.
        However, the cast and crew columns are not composed of discrete datapoints,
            instead being json files representing the entire cast or crew list.
        As such, when working with it, we had to extract the data using string parsing
            methods.
        The column summary is shown in Table \ref{tab-Credits-Column-Description}.
        \begin{table}[h]
            \centering
            \begin{tabular}{lp{10cm}l}
                \toprule
                Column Name & Description                                                              & Data Type  \\
                \midrule
                cast        & The cast list of the movie, including all actors who appeared in it.     & .json file \\
                crew        & The entire crew list of the movie, including all the people who made it. & .json file \\
                id          & The movie id - used in the larger dataset to connect tables together     & Integer    \\
                \bottomrule
            \end{tabular}
            \caption[short]{The different columns in the credits.csv file}\label{tab-Credits-Column-Description}
        \end{table}

    \paragraph{Data processing}
        To assist in parsing the TMD dataset, we used a Python script to count the
            amount of movies each individual director or actor has helped make.
        The resulting datasets were saved as CSV files named actor\_counts.csv and
            director\_counts.csv respectively.
        This newly acquired data was then merged with the original IMDb dataset, which
            involved dropping the Description column while replacing the Director and Actor
            columns with Director Exp.
        and Mean Lead Roles Exp.
        This resulted in a merged data set with structure shown in Table
            \ref{tab-merged-data-column-description}.
        \begin{table}[h]
            \begin{tabular}{lp{9cm}l}
                \toprule
                Column Name          & Description                                                  & Data Type \\
                \midrule
                Rank                 & See table \ref{tab-IMDb-Movie-Data-Column-Description}       & Integer   \\
                Title                & See table \ref{tab-IMDb-Movie-Data-Column-Description}       & String    \\
                Genre                & See table \ref{tab-IMDb-Movie-Data-Column-Description}       & Genre     \\
                Director Exp.        & The number of movies that the director of the movie has made & Float     \\
                Mean Lead Roles Exp. & The mean number of movies that the lead actors have been in  & Float     \\
                Year                 & See table \ref{tab-IMDb-Movie-Data-Column-Description}       & Integer   \\
                Runtime (Minutes)    & See table \ref{tab-IMDb-Movie-Data-Column-Description}       & Integer   \\
                Rating               & See table \ref{tab-IMDb-Movie-Data-Column-Description}       & Float     \\
                Votes                & See table \ref{tab-IMDb-Movie-Data-Column-Description}       & Integer   \\
                Revenue (Millions)   & See table \ref{tab-IMDb-Movie-Data-Column-Description}       & Float     \\
                Metascore            & See table \ref{tab-IMDb-Movie-Data-Column-Description}       & Integer   \\
                \bottomrule
            \end{tabular}
            \caption[short]{The different columns in the merged data set}\label{tab-merged-data-column-description}
        \end{table}

        To check this data was properly normalised we made a histogram plot of all the
            numeric variables, shown in Figure \ref{fig-distribution-of-numeric-variable}.
        As expected, a few variables did not appear to be normally distributed, namely:
        Revenue (Millions), Votes, Runtime (Minutes), Director Exp., and Rating.
        \begin{figure}[H]
            \centering
            \includegraphics[width=0.8\linewidth]{Final/Distribution of Numeric Variables (No Transformation).pdf}
            \caption[short]{The distributions of the numeric variables in the merged dataset}\label{fig-distribution-of-numeric-variable}
        \end{figure}
        As shown in the plot, Revenue (Millions) and Votes are severly right skewed,
            implying an exponential distribution; Runtime (Minutes) and Director Exp.
        appear to be less severly right-skewed, implying a lognormal distribution;
        Rating seems to be left-skewed.
        The transformations for these variables that gave the best approximations to a
            normal distribution were:
        \begin{multicols}{2}
            \begin{itemize}
                \item Revenue (Millions) : Cube Root transform
                \item Votes              : Cube Root transform
                \item Runtime (Minutes)  : Log transform
                \item Director Exp.      : Log transform
                \item Rating             : Square transform
            \end{itemize}
        \end{multicols}
        Figure \ref{fig-transformed-distribution-of-numeric-variable} shows the
            transformed and normalised numeric variables.
        The label has the p-values from testing whether the transformed distribution is
            normal, using the Kolmogorov-Smirnov test\cite{KStest} for goodness of fit.
        One interesting note is that altough the Director Exp.
        column fails the Kolmogorov-Smirnov test, there is clearly missing
        data; around 3 columns are missing from the histogram shown.
        With this in mind, and as the histogram does follow the normal distribution
            curve, there is sufficient evidence to assume ln(Director Exp) has a normal
            distribution.
        \begin{figure}[H]
        \centering
        \includegraphics[width=\linewidth]{Final/Distribution of Numeric Variables (Transformed).pdf}
        \caption[short]{The distributions of the standardised and normalised numeric variables in the merged dataset,
        the legend has the p-values from testing with $H_{0}: X \not\sim N(0,1)$.
        Also shown on the plots is the Gaussian Distribution with the columns mean and
            standard deviation}\label{fig-transformed-distribution-of-numeric-variable}
            \end{figure} Figure \ref{fig-transformed-distribution-of-numeric-variable}
            shows the transformed and normalised numeric variables.
        The legend contains the p-values from testing the data with $H_{0}: X \not\sim
                N(0,1)$, using the Kolmogorov-Smirnov normality test\cite{KStest} for goodness
            of fit.
        It's worth noting that even though the Director Exp.
        column failed the Kolmogorov-Smirnov test, there is missing data present in the histogram shown.
        Given that the histogram follows a normal distribution curve, there is
            convincing evidence to assume ln(Director Exp.
        ) has a normal distribution.
        Overall, these transformations helped to normalize the dataset and enhance its
            suitability for further analysis.
